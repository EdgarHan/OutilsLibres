
\begin{document}
Maître Corbeau, sur un arbre perchai,
Tenais en son bec un fromage.
Maître Renard, par l'odeur alléché,
Lui tint à peu près ce langage :
"Hé ! bonjour, Monsieur du Corbeau.
Que vous êtes joli ! que vous me semblez beau !
Sans mentir, si votre ramage
Se rapporte à votre plumage,
Vous êtes le Phénix des hôtes de ces bois. "
A ces mots le Corbeau ne se sent pas de joie ;
Et pour montrer sa belle voix,
Il ouvre un large bec, laisse tomber sa proie.
Le Renard s'en saisie, et dis : "Mon bon Monsieur,
Apprenez que tout flatteur
Vit aux dépens de celui qui l'écoute :
Cette leçon vaut bien un fromage, sans doute. "
Le Corbeau, honteux et confus,
Jura, mais un peu tard, qu'on ne l'y prendrait plus.
